%%%%%%%%%%%%%%%%%%%%%%%%%%%%%%%%%%%%%%%%%
% University/School Laboratory Report
% LaTeX Template
% Version 3.0 (4/2/13) %
% This template has been downloaded from:
% http://www.LaTeXTemplates.com
%
% Original author:
% Linux and Unix Users Group at Virginia Tech Wiki 
% (https://vtluug.org/wiki/Example_LaTeX_chem_lab_report)
%
% License:
% CC BY-NC-SA 3.0 (http://creativecommons.org/licenses/by-nc-sa/3.0/)
%
%%%%%%%%%%%%%%%%%%%%%%%%%%%%%%%%%%%%%%%%%

%----------------------------------------------------------------------------------------
%	PACKAGES AND DOCUMENT CONFIGURATIONS
%----------------------------------------------------------------------------------------

\documentclass{article}
\usepackage{graphicx} % Required for the inclusion of images
\usepackage{hyperref}
%\setlength\parindent{0pt} % Removes all indentation from paragraphs

\renewcommand{\labelenumi}{\alph{enumi}.} % Make numbering in the enumerate environment by letter rather than number (e.g. section 6)

%\usepackage{times} % Uncomment to use the Times New Roman font

%----------------------------------------------------------------------------------------
%	DOCUMENT INFORMATION
%----------------------------------------------------------------------------------------

\title{How To Use The ITR Card Scanner}

\author{Peter \textsc{Chinetti}} % Author name

\begin{document}

\maketitle % Insert the title, author and date

\section{Overview}
This card scanner allows you to advertise your presence in the Robotics Lab by tapping your IIT ID. The first time you tap your card, it will present a screen for you to enter a name and take a photo with the webcam. It will also check you into the lab. Successive taps will alternate between checking you in and out, and inform you of your status.\\
See who is in the lab by going to \url{illinoistechrobotics.org/members.html}, which is updated every minute.

\section{First Usage}
\begin{enumerate}
\item Scan your card.
\item A text box will pop up. Fill in the \textsc{Name} field.
\item After filling in the \textsc{Name} field, click the \textsc{Take Photo} button. The system will wait for 5 seconds before capturing the image. 
  \begin{itemize}
  \item If the first picture is bad, you can take another by pressing the button again
  \item Your name is captured when you click \textsc{Take Photo}, so if you'd like to correct it, take another photo.
    \item The box on the far wall in brown tape marks the edges of the camera's range. Use it to frame your photos. (I'm 6'6'', so it's pretty tall for many people. I put a chair in here to stand on).
  \end{itemize}
  \item Press \textsc{Submit} to save your photo and check yourself in.
\end{enumerate}
\section{Successive Usages}
Just tap your card. A window will pop up with your name, your card ID number, and whether you've been checked in or out.
\end{document}
